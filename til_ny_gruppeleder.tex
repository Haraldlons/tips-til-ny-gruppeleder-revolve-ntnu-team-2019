\documentclass[11pt]{article}
\usepackage[english]{babel}   % English names on Introduction and other places
% \usepackage[norsk]{babel}   % Norwegian names on Introduction and other places
\usepackage[T1]{fontenc}        % Norwegian charset tegnsett (æøå)
\usepackage[utf8]{inputenc} % Norwegian charset
\usepackage{geometry}       % Recommended package for controlling margins.
\usepackage{natbib}
\usepackage{amsmath}
\usepackage{caption}
\usepackage{amssymb} %Comments here are fine
\usepackage{float}
\usepackage{lmodern}
\usepackage{parskip}
\usepackage{textcomp}
\usepackage{booktabs}
\usepackage{graphicx}

\usepackage{hyperref}
\hypersetup{
    colorlinks=true,
    linkcolor=blue, % Color of links in 'innholdsfortegnelse'
    filecolor=blue, % Doesn't show when colorlinks=true. It's the border-color.
    urlcolor=blue, % Links wil be nice blue color
}
\newcommand\tab[1][1cm]{\hspace*{#1}}

\usepackage{listings}
\lstset{basicstyle=\ttfamily,
  showstringspaces=false,
  commentstyle=\color{red},
  keywordstyle=\color{blue},
  escapeinside={(*@}{@*)},          % if you want to add LaTeX within your code
}


\begin{document}

\title{\textbf{Forlag til arbeidsoppgaver til ny gruppeleder \\+ noen tips}}
\author{Author: Harald Lønsethagen}
% \date{\today}
\date{Last edited: \today\tab Version: 1.0}

\maketitle


% \clearpage\thispagestyle{empty}
\makeatletter
%\renewcommand\tableofcontents{\@starttoc{toc}}
%\makeatother
\tableofcontents
\newpage
\section{Motivasjon}
Det kan være vanskelig å vite hvor man skal starte å jobbe når man er en ny gruppeleder. Jeg husker godt da jeg selv var ny, og det var mye å sette seg inn i. Jeg syntes det var vanskelig å vite hva som er relevant å lese seg opp på. Derfor, som en hjelp til kommende gruppeleder(e) lager jeg dette dokumentet. 
\subsection*{Forord}
Dette er mine personlige meninger, etter over et år som gruppeleder for \textbf{Guidance \& Control}, der jeg har gjort mange feil og gått i mange feller. Ta det jeg skriver med en klype salt. Og for all del, dette er kun \textbf{forslag} til arbeidsoppgaver.

\section{Tekniske arbeidsoppgaver}
Her skriver jeg litt om hva som kan være lurt å gjøre. Blir en skikkelig smørbrødliste, så det vil ta lang tid å bli kjent med alle områdene jeg presenterer.

% \subsection{Softwareutvikling i store team}
% Les deg opp på
% \begin{itemize}
% 	\item TDD
% 	\item Git
% 	\item Git Submodules
% 	\item 
% \end{itemize}

\begin{itemize}
	\item Videos
	\begin{itemize}
		\item \href{https://vimeo.com/41854831}{Claude Rouelle: Advice for SAE Teams}
		\item \href{https://vimeo.com/236115599}{ROSCon 2017 Vancouver Day 1: Autonomous Racing Car for Formula Student Driverless}
		\item Ta en titt på \href{https://www.youtube.com/watch?v=FbKLE7uar9Y&list=PLzugcz36-zVEIKpvXdxbok9p3uxZvYYFN}{denne} spillelisten med FSD-videoer
	\end{itemize}
	\item ROS
	\item Lese ADR til AMZ (eller det som ligner på ADR)
	\item Laste ned Ubuntu, installere ROS ++
	\item Generelt, let rundt etter artikler om SW-utvikling i store prosjekter. \\ \tab
	Disclaimer: Det er vanskeligere enn du tror :P 
	\item \href{https://medium.com/@porteneuve/mastering-git-submodules-34c65e940407}{Git Submodules}
	\item Les seg opp på hvorfor modularisering er viktig
	\item Les seg opp på hvorfor automatiske tester er viktig
	\item Test Driven Development (TDD)

\end{itemize}

\section{Organisatoriske arbeidsoppgaver}
Dette er tips til arbeidsoppgaver som handler om det organisatoriske. 

\begin{itemize}
	\item Lese grupperefleksjoner
	\item Lese forkortet versjon av reglene
	\item Skrive ned spørsmål du har om Revolve i \href{http://revolvevault.ivt.ntnu.no:8090/display/GEN/Frequently+asked+questions+about+Revolve}{FAQ-siden}
	\item Reflektere litt rundt hva slags gruppe du ønsker å danne
	\item Reflektere hvordan gruppeleder du ønsker å være
	\item Snakk med andre gruppeledere og hør hvilke erfaringer de har
	\item Gjør deg kjent på Confluence\\ \tab
	Gjerne lag masse sider og utforsk hvilke muligheter som finnes
	\item Tenk på hvordan dere kan opprettholde struktur og orden i software. Er du en strukturert person som kan hjelpe til med dette, eller trenger gruppen en dedikert person til det?
	\item Skriv ned hva du anser som ansvarsområde til gruppen din, og diskuter med andre gruppeledere om dere er samstemte.
	\item Les erfaringsoverføringen til Rebecca.
	\item Les erfaringsoverføringen til Cornelia.
	\item Les flere erfaringsoverføringer. Du vil mest sannsynlig finne enkelte punkter som går igjen.
	\item Lag deg et dokument som skal bli ditt lille barn. Dette dokumentet skal inneholde forbedringer og ideer til arbeidet i Revolve. Hver gang du får en ide om at noe kan gjøres bedre, så skriver du det inn her, så er du sikker på at du har et sentralt sted du vet du har skrevet ned ideene dine. 
	\item Skum gjennom det som finnes på Google Drive. Spør folk hva de syntes om Drive/Confluence.
	\item Spør folk generelt veldig mye. Vær frampå. Innse at dette vervet krever at du er veldig PROAKTIV. Hvis noe ikke er så bra som du selv mener det skal være, så må du skjønne at du SELV må få det til å bli bedre. Her kan du ikke bare klage, men aksjonere! Hva mener du ikke er godt nok? Hvordan skal det fikses? Hvilke tiltak kan iverksettes for å forbedre dem? Er de realistiske?
\end{itemize}
% References

\section{Generelle tips}
\subsection{Fake it until you make it}
Skjønn at du ikke er så erfaren. Innse at du har mye å lære. Men, lat som du er den gode gruppelederen du ønsker å være. Stå stødig, rett rygg. 
\subsection{Muligheter og komfortsone}
Innse hvor utrolig store muligheter du har foran deg, og at du må gripe dem.  
Prøv å utfordre deg selv med å gå utenfor komfortsonen sin. 
Dette er også en fantastisk kul måte å kunne utfordre seg selv på.

\subsection{Lederstil og ærlighet}
Som jeg tidligere har nevnt tror jeg det er lurt om du reflekterer litt over hvordan type leder du ønsker å være. Hvordan miljø ønsker du at gruppa din skal ha det, og hvordan kan du påvirke dem til å bli det du ønsker? Hvis ting ikke går akkurat slik du ønsker, hva skal du da gjøre? Hvis det oppstår konflikter som kan være vanskelig å løse, hvordan skal man håndtere dem? Skal man spørre om hjelp, håndtere det selv, være streng, snill, kompis eller autoritær? Det finnes på ingen måte noe fasitsvar, men ikke forvent at alt skal gå på skinner, for hvis det er en ting jeg har lært i Revolve, så er det at det alltid oppstår 'situasjoner'.\\
\\
Dog vil jeg gi deg et tips som handler om hvordan forhold man skal ha til sine medlemmer. Jeg har hele tiden prøvd å ha et ganske ærlig forhold. Med ærlig mener jeg at jeg skal si ifra til dem når de gjør de bra, men jeg skal også si ifra, direkte og konkret, hva slags oppførsel jeg ikke syntes holder standard. Dette er på ingen måte enkelt, og jeg har begått så mange tabber selv. Men det jeg har lært, er at hvis du ikke er respektløs, men viser at du oppriktig bryr deg om medlemmene dine og vil at de skal lykkes, så kan du gi dem ganske så ærlige tilbakemeldinger, og de vil ta det veldig godt. På ingen måte gå til personangrep, men forklar hvilke \textbf{handlinger} som du ikke akseptere, og hvordan du tenker. Jeg hadde mye suksess med å si høyt hva jeg selv tenker når medlemmene gjør ting jeg ikke aksepterer. Her kommer et eksempel hvordan jeg ville formulert meg til et medlem i en hypotetisk situasjon: \textit{Når du kommer stadig sent, og ikke sier ifra, får jeg en følelse av at jeg ikke har kontroll over hvor du er eller om du kommer til å møte opp. Dette er veldig uheldig, da jeg må bruke mye tid på å følge deg opp, og jeg må på en måte passe på deg.}\footnote{Teksten er oppdiktet for å gi et eksempel på hvordan man kan formulere seg} \\

Min erfaring med lignende scenarioer er at medlemmene er forståelsesfulle og er meget villige til å forbedre situasjonen. Folk liker som regel å forbedre seg, så om du kan fasilitere selvutvikling er slike samtaler gull verdt. \\

Det jeg har lært meg er at alt dette med lederskap og hvordan håndtering av ulike situasjon, alt dette er en treningssak. Tidligere hadde jeg en fordom om at man var 'født' en leder eller ikke. Dette har jeg så til de grader skjønt ikke stemmer. Lederskap er som det meste annet i livet... det kan trenes opp. 

\subsection{Et givende år}
Noe av det jeg personlig syntes var mest givende med året var at jeg hadde så mye frihet til å gjøre egne ting. Hadde jeg en idé kunne jeg bare planlegge, iverksette og sjekke om det fungerte. F.eks. hadde jeg mange ideer om hvordan man kunne motivere en gruppe. Jeg derfor bare hoppet ut i det og gjorde litt 'teite' ting, men egentlig syntes medlemmene det bare var bra. \\
	F.eks. så lagde jeg mye system på Confluence, og syntes det var givende å være med å lage standarer.
	Andre ting jeg 'bare' gjorde:
\begin{itemize}
	\item Stå mens jeg hadde motivasjonstaler
	\item Quiz-script
	\item Automatiske maler
	\item Bordtennis-liga
\end{itemize}

\subsection*{Siste ord}
Er det noe du lurer på, så er jeg mer enn villig til å hjelpe til. Ta kontakt hvor som helst. \\
Dette dokumentet finnes i Git-repoet her: \\ \url{https://github.com/Haraldlons/tips-til-ny-gruppeleder-revolve-ntnu-team-2019.git}\\

Helt til slutt. Gleg deg! Jeg er sikker på at dette kommer til å bli et veldig bra år!


% \bibliographystyle{plain}
% \bibliography{references}

\end{document}
